%%%%%%%%%%%%%%%%%%%%%%%%%%%%%%%%%%%%%%%%%
% * <rjcsvv@gmail.com> 2017-03-31T16:40:18.072Z:
%
% ^.
% University Assignment Title Page 
% LaTeX Template
% Version 1.0 (27/12/12)
%
% This template has been downloaded from:
% http://www.LaTeXTemplates.com
%
% Original author:
% WikiBooks (http://en.wikibooks.org/wiki/LaTeX/Title_Creation)
%
% License:
% CC BY-NC-SA 3.0 (http://creativecommons.org/licenses/by-nc-sa/3.0/)
% 
% Instructions for using this template:
% This title page is capable of being compiled as is. This is not useful for 
% including it in another document. To do this, you have two options: 
%
% 1) Copy/paste everything between \begin{document} and \end{document} 
% starting at \begin{titlepage} and paste this into another LaTeX file where you 
% want your title page.
% OR
% 2) Remove everything outside the \begin{titlepage} and \end{titlepage} and 
% move this file to the same directory as the LaTeX file you wish to add it to. 
% Then add \input{./title_page_1.tex} to your LaTeX file where you want your
% title page.
%
%%%%%%%%%%%%%%%%%%%%%%%%%%%%%%%%%%%%%%%%%
%\title{Title page with logo}
%----------------------------------------------------------------------------------------
%	PACKAGES AND OTHER DOCUMENT CONFIGURATIONS
%----------------------------------------------------------------------------------------

\documentclass[12pt]{article}
\usepackage{amsmath}
\usepackage{graphicx}
\usepackage[colorinlistoftodos]{todonotes}
\usepackage[utf8x]{inputenc}
\usepackage[portuguese]{babel}
\usepackage{indentfirst}
\usepackage{float}
\usepackage{listings}

\begin{document}

\begin{titlepage}

\newcommand{\HRule}{\rule{\linewidth}{0.5mm}} % Defines a new command for the horizontal lines, change thickness here

\center % Center everything on the page
 
%----------------------------------------------------------------------------------------
%	HEADING SECTIONS
%----------------------------------------------------------------------------------------

\textsc{\LARGE Universidade do Minho}\\[1.5cm] % Name of your university/college
\textsc{\Large Mestrado Integrado em Engenharia Informática}\\[0.5cm] % Major heading such as course name
\textsc{\large Computação Gráfica}\\[0.5cm] % Minor heading such as course title

%----------------------------------------------------------------------------------------
%	TITLE SECTION
%----------------------------------------------------------------------------------------

\textbf\HRule \\[0.4cm]
{ \huge \bfseries Fase 2 - Sistema Solar Estático}\\[0.4cm] % Title of your document
\HRule \\[1.0cm]
 
%----------------------------------------------------------------------------------------
%	AUTHOR SECTION
%----------------------------------------------------------------------------------------

\begin{minipage}{0.4\textwidth}
\begin{flushleft} \large
André Germano \\
André Freitas \\
Sofia Carvalho \\
\end{flushleft}
\end{minipage}
~
\begin{minipage}{0.4\textwidth}
\begin{flushright} \large
A71150 \\
A74619 \\
A76658 \\
\end{flushright}
\end{minipage}\\[1.0cm]

% If you don't want a supervisor, uncomment the two lines below and remove the section above
%\Large \emph{Author:}\\
%John \textsc{Smith}\\[3cm] % Your name

%----------------------------------------------------------------------------------------
%	LOGO SECTION
%----------------------------------------------------------------------------------------
%----------------------------------------------------------------------------------------
\newpage % Fill the rest of the page with whitespace

\end{titlepage}

\tableofcontents

\begin{appendix}
	\listoffigures
    \renewcommand\lstlistlistingname{Listagens}
    \lstlistoflistings
\end{appendix}

\vfill
\newpage
\section{Implementação}
De forma a facilitar a compreensão do projeto, foi decidido dividir em três partes distintas tendo em conta o objetivo do projeto.

Para facilitar a leitura e análise das soluções, o \emph{stdout} do analisador léxico é alterado para um apontador de \emph{FILE}.
\begin{center}
\begin{minipage}{8cm}
\begin{verbatim}
int main(int argc,char *argv[]){
	yyin=fopen(argv[1],"r");
	yyout=fopen("...","w");
	yylex();
	return 0;
}
\end{verbatim}
\end{minipage}
\end{center}

\subsection{Acentos Explícitos e Indentação}
Começamos então por converter todos os caracteres com acentos explícitos em caracteres portugueses, trocar todas as aspas por chavetas e tornar a indentação do texto mais apelativa.

De forma a facilitar a escrita de expressões regulares foram elaboradas as seguintes definições:
\begin{center}
\begin{minipage}{8cm}
\begin{verbatim}
a1 	        \{\\’a\}
a2 	        \{\\`a\}
...
u1 	        \{\\’u\}
u2 	        \{\\’U\}
c           \{\\c\{c\}\}

D           de|da|dos|do
lm 			        [a-z]|\.|{acent}|{pt}
P           ([A-Z]|{acentM}|{pt}){lm}+
nome        {P}([ ]({P}|{D}))*  
author      (AUTHOR|author)[ ]*=[ ]*
editor      (EDITOR|editor)[ ]*=[ ]*
pt          {a1}|{a2}|{a3}|{a4}|{a5}|{e1}|
{e2}|{i1}|{i2}|{o1}|{o2}|{o3}|{u1}|{u2}|{c}
acent       á|à|ã|Á|À|é|É|í|Í|ó|õ|Ó|ú|Ú|ç
acentM      Á|À|É|Í|Ó|Ú      
char        [a-zA-z]|{acent}|\.
\end{verbatim}
\end{minipage}
\end{center}

\newpage

Começámos por obter todas linhas cujos campos são \emph{author} ou \emph{editor}, para isso foram utilizadas as seguintes regras respetivamente:
\begin{verbatim}
<*>{author}(\{|\")  {BEGIN TOKENS;fprintf(yyout,"%s","author = {");}
<*>{editor}(\{|\")  {BEGIN TOKENS;fprintf(yyout,"%s","editor = {");}
\end{verbatim}

Sempre que for encontrado um campo \emph{author} ou \emph{editor} são invocadas um conjunto de regras cuja \emph{Start Condition} é \emph{TOKENS}, regras essas que permitem efetuar o \emph{parsing} da linha desejada e remover todos os acentos explícitos,tal como efetuar a indentação pretendida.
\begin{verbatim}
<TOKENS>[ ]*(\}|\")\,    {fprintf(yyout,"%s","},");BEGIN(0);}
<TOKENS>{a1}             {fprintf(yyout,"%s","á");}
<TOKENS>{a2}             {fprintf(yyout,"%s","à");}
<TOKENS>{a3}             {fprintf(yyout,"%s","ã");}
...
\end{verbatim}

A primeira das regras apresentadas anteriormente permite identificar o final da linha e portanto, através da \emph{Star Condition BEGIN(0)} retornar ao estado inicial, procurando o próximo campo pretendido.

O input dado como exemplo no enunciado produz o seguinte output:

\begin{verbatim}
author = {Ricardo Martini and Cristiana Araújo and Pedro Rangel Henriques},
editor = {Giovani Librelotto},
...
author = {Vilas Boas, Ismael and Oliveira, Nuno and Henriques, Pedro Rangel},
...
\end{verbatim}

\newpage
\subsection{Nomes e Sobrenomes}
Para a segunda parte do projeto, é requerido que um autor seja identificado pelo Apelido seguido de uma vírgula e letras iniciais dos restantes nomes seguidas por um '.'.

De forma a atingir este fim, foi necessário, nas linhas com campos \emph{autor} e \emph{editor}, ter em atenção que poderão já haver Apelidos seguidos de vírgulas, sendo que neste caso apenas é necessário analisar e gerar os restantes nomes. 

Foram então implementadas as seguintes regras:
\begin{verbatim}
<AUTORES>{nome}\,	          {
                           fprintf(yyout,"%s",yytext);
                           BEGIN ACASE;
                           }
                           
<AUTORES>[A-Z]|{acentM}    {
                           strcpy(sobrenome,"\0");
                           strcat(sobrenome,yytext);
                           strcat(sobrenome,".");
                           BEGIN BCASE;
                           }
\end{verbatim}

No primeiro caso, são invocadas as regras com \emph{Start Condition ACASE}.
\begin{verbatim}
<ACASE>[A-Z]|{acentM}     {fprintf(yyout,"%s.",yytext);}
<ACASE>{lm}*|\.           {}
<ACASE>and[ ]             {fprintf(yyout,"%s",yytext);BEGIN AUTORES;}
<ACASE>\}                 {fprintf(yyout,"%s",yytext);BEGIN(0);}
\end{verbatim}

Partindo do princípio de que todos os nomes começam com letra maiúscula, conseguimos obter a letra inicial de cada nome através da primeira regra, concatenando de seguida o '.' e removendo os restantes caracteres pela segunda regra.

Uma vez que o \emph{and} separa os diferentes autores ou editores, sempre que o analisador gráfico o encontra, são novamente invocadas as regras do \emph{Start Condition AUTORES}, isto aconte até à ocorrência do caracter \} que determina o final de linha, neste caso o analisador léxico volta à condição inicial.

O segundo caso, é um pouco mais complexo, assim sendo foi necessário recorrer à utilização de duas variáveis globais(nome e sobrenome) que permitem guardar a informação necessária para processar o nome dos autores,complementadas com as seguintes regras:
\begin{verbatim}
<BCASE>[A-Z]|{acentM}|([ ]{D}[ ])   {
                                    strcat(sobrenome," ");
                                    strcat(sobrenome,yytext);
                                    strcat(sobrenome,".");
                                    }
                                    
<BCASE>{P}[ ]and                    {
                                    yytext[yyleng-4]='\0';
                                    nome=strdup(yytext);
                                    fprintf(yyout,"%s, %s and",nome,sobrenome);
                                    BEGIN AUTORES;
                                    }
                                    
<BCASE>{P}\}                        {
                                    yytext[yyleng-1]='\0';  <-ação 1
                                    nome=strdup(yytext);
                                    fprintf(yyout,"%s, %s}",nome,sobrenome);
                                    BEGIN(0);
                                    }
                                    
<BCASE>{lm}*[ ]|\.                  {} 
\end{verbatim}

Foram tidos em conta dois casos em relação aos nomes. 
Quando um nome é seguido do caracter '\}', estamos perante o final de linha, sendo '\}' retirado da string \emph{yytext} através da ação 1, de seguida o analisador léxico retorna novamente às regras iniciais.

Quando um nome é seguido por um \emph{and}, sabemos então que existem mais nomes e portanto para além de processar os nome e sobrenomes já guardados nas respetivas variáveis globais, o analisador léxico volta às regras da \emph{Start Condition AUTORES}.

Em ambos os casos, caso um nome não seja seguido de \emph{and} ou \emph{\{} trata-se de um sobrenome, sobrenome esse que é concatenado à uma variável global \emph{sobrenome} ter sido concatenado o caracter '.' ao primeiro caracter. Todos os outros caracteres são removidos através da última regra.

\subsection{Grafo}
Para finalizar, foi requerida a construção de um grafo com todas as colaborações entre autores. Deste modo, utilizámos quatro variáveis globais(autores,colabora,totalColab e numero).
\begin{center}
\begin{minipage}{8cm}
\begin{verbatim}
char *autores[1024];
	char *colabora[1024];
	int totalColab[1024];
	int numero;
\end{verbatim}
\end{minipage}
\end{center}

Foram considerados duas situações. Na primeira situação, um nome pode ser sucedido de \emph{and}, o que significa que existe um ou mais colaboradores e a segunda situação que que um nome é sucedido por \emph{'\}'}.

Caso um colaborador seja seguido de \emph{and}, é adicionado na variável global \emph{char *autores[]}, na posição numero, que corresponde ao total de autores encontrados até ao momento. Caso um colaborador seja seguido de \emph{'\}'}, é também adicionado à variável \emph{autores} e é feito o processamento dessa mesma variável através da função \textbf{\emph{char **processa()}}, que retorna um \emph{array de strings} contendo todas as combinações(colaborações) entre os autores recebidos como argumento, combinações essas que através da função \textbf{\emph{insSRepetidos()}} vão ser inseridas, sem repetições à variável \emph{char **colabora}. Para além disso é também calculado o número total de colaborações entre autores.
\begin{verbatim}
<AUTORES>{nome}\,([ ]{abv})*[ ]and  
                  {
                  yytext[yyleng-4]='\0';
                  autores[numero]=strdup(yytext);
                  numero++;
                  }
<AUTORES>{nome}\,([ ]{abv})*\}      
                  {
                  yytext[yyleng-1]='\0';
                  autores[numero]=strdup(yytext);
                  if(numero>=1){
                  	char **aut=processa(autores,numero);
                    insSRepetidos(colabora,aut,totalColab);
                  }
                  BEGIN(0);
\end{verbatim}

Finalmente, quando o analisador léxico chega ao final do ficheiro são imprimidas, num ficheiro com a extensão \emph{.dot} todas as colaborações entre autores, através da função \textbf{\emph{printGraph()}}.
\begin{verbatim}
<*><<EOF>>       {printGraph(colabora,totalColab);return 0;}
\end{verbatim}

O output gerado pela função \textbf{\emph{printGraph()}} é o seguinte:
\begin{verbatim}
digraph Colaboradores{
   size="15,15";
   rankdir=LR;
   "Almeida, J. J." -> "Martins, F." [label="96"];
   "Henriques, P. R." -> "Martins, F." [label="14"];
   "Almeida, J. J." -> "Henriques, P. R." [label="10"];
   "Almeida, J. J." -> "Barros, J. B." [label="7"];
   "Almeida, J. J." -> "Pinto, U." [label="2"];
   "Almeida, J. J." -> "Barbosa, L. S." [label="5"];
   "Almeida, J. J." -> "Ramalho, J. C." [label="1"];
   "Henriques, P. R." -> "Ramalho, J. C." [label="1"];
   "Nijholt, A." -> "Scollo, G." [label="1"];
   ...
   }
\end{verbatim}

Em que cada nó representa um colaborador e cada ligação entre nós contém o total de colaborações entre esses dois autores.
\end{document}